% langLearnVar 
% Annual Cognitive Science Conference 2016

\documentclass[10pt,letterpaper]{article}

\usepackage{cogsci}
\usepackage{pslatex}
\usepackage{apacite2}
\usepackage{graphicx}
\usepackage{array,multirow}



\newcommand*\rot{\rotatebox{90}}
\newcommand{\squeezeup}{\vspace{-1.5mm}}

\title{Learnability pressures on language across multiple timescales}
\author{{\large \bf Molly Lewis} \\ \texttt{mll@stanford.edu}\\ Department of Psychology \\ Stanford University \\ 
\And {\large \bf Michael C. Frank} \\ \texttt{mcfrank@stanford.edu} \\ Department of Psychology \\ Stanford University \\ }

\begin{document}

\maketitle

\begin{abstract}


\textbf{Keywords:} 
Linguistic Niche Hypothesis; language evolution
\end{abstract}


%%%%% INTRODUCTION %%%%%%
\section{Introduction}
Many facts about the social world are explicable only by considering the broader context: Why do gloves have five fingers? Why do tropical countries use lots of spices? The answers to these questions rely on taking into account properties of individual humans (hands have five fingers), and properties of the broader environmental context  \cite<tropical countries are hot, and spices kill bacteria;>{sherman1999darwinian}. A growing body of work has begun to also explain {\it language structure} by  taking into account these same contextual factors. This work argues that language systems are the product of cognitive constraints internal to the language learner \cite{chater2010language}, as well as properties of the environmental context \cite{nettle2012social} . 

In this paper, we explore this second proposal---that properties of the environmental context shape language systems. This hypothesis, what has been termed the  {\it Linguistic Niche Hypothesis} \cite{lupyan2010language,wray2007consequences}, suggests that language systems adapt to the pressures in the environment where a language is spoken. These pressures may come from a range of sources, including the climate, the length of the growing season, or the demographic characteristics of the people who learn and speak the language. 

An important aspect of this hypothesis is the scale over which these pressures operate \cite{mcmurray2012,blythe2015hierarchy}. There are at least two scales at which language can be studied: individual utterances, which change over the course of moments in a communicative interaction, and language systems, which change over  the timescale of years.  An established body of work supports the claim  thats speakers adapt to the context at the scale of individual utterances, both in terms of the properties of the listener and the physical environment \cite<e.g.,>{schober1993spatial,frank2012predicting}. The present hypothesis explores the more controversial claim that environmental pressures shape language at the scale of language systems. %2x2 figure?

There are a number of pieces of evidence that environmental factors may indeed shape language systems.  At the lowest level of the linguistic hierarchy, languages with larger populations are claimed to have larger phonemic inventories \cite{atkinson2011phonemic,hay2007phoneme}, but  shorter words \cite{wichmann2011phonological}. Speakers with more second language learners have also been suggested to have a lower type-token ratio of lexical items \cite{bentz2015adaptive}. At the level of morphology, evidence suggests that speakers with larger populations tend to have simpler morphology \cite{lupyan2010language, bentz2013languages}. Finally, there is also evidence that population size may influence the mappings between form and meaning. In particular, this work suggests that languages tend to map longer words to more complex meanings \cite{lewisstructure2014}, but that this bias is smaller for languages with larger populations  \cite{lewis_evolang}.

The plausibility of the Linguistic Niche Hypothesis depends largely on the presence of a possible mechanism linking environmental features to aspects of language systems. A range of proposals have been suggested \cite{nettle2012social}. For example, one possibility is that children (L1) and adult (L2) language-learners differ in their learning constraints. In particular, children may be better at acquiring complex morphology than adults, and so languages with mostly children learners may tend to have more complex morphology. A second possibility is that speakers in with less dense social networks have less variable linguistic input, and this leads the language system to have more complex morphology.  

%The reported relationships between environmental features and linguistic features described above are efforts to test these hypotheses using available proxies.

Testing these mechanisms is empirically challenging, however. Because there are many factors that shape a linguistic system, large datasets are needed to detect an effect of environmental factors. In addition, many languages are non-independent because of genetic relationships and language contact, and so data from a wide range of languages are needed to control for these moderators. Third, the scale of this hypothesis makes it difficult to directly intervene on the mechanism. Finally, the hypothesized mechanisms are somewhat underspecified, and the dynamics of these different factors may be complex, trading-off with each other in non-obvious ways \cite<e.g.>{wichmann2011phonological}.


In this work, we try to address these challenges by clarifying the empirical landscape. We do this by aggregating across datasets that find covariation between environmental variables and linguistic structure. This serves two purposes. First, it allows us to examine the relationship between the same set of environmental predictors across a range of linguistic features. And, second, it allows for the same analytical techniques and areal controls to be used across datasets. By addressing these inconsistencies, we are in a better position to more directly compare relationships between environmental and linguistic features. Importantly, a more coherent picture of the empirical landscape may provide insight into the mechanism linking language systems to their environments.

We also more directly address the question of mechanism by examining variability in the mean age of acquisition of words for L1 learners across languages. Evidence that this variability is related to an aspect of the linguistic system (such as number of phonemes) would suggest that L1 learners, and not L2 learners, are the relevant environmental factor shaping that aspect of the linguistic system. 

In what follows, we first present a set of analyses examining the relationship between environmental and linguistic features using the same analytical techniques (Analysis 1). In Analysis 2, we examine the relationship between a languages mean age of acquistion and aspects of the linguistic system.

 
%%%% Demo and language features%%%%%%
\section{Analysis 1: Environmental pressures on language systems}
\cite{jaeger2011language}
\cite{moran2012revisiting}

%%%% AOA %%%%%%
\section{Analysis 2: L1 learning and language systems}
\cite{luniewska2015ratings}

% of the linguistic system. If languages with certain Evidence for a relationship for L1 learners would be suggestive that the cognitive constraints of the 


 Other work has empirical + modeling
Some attempts
\cite{silvey2015word}
\cite{perfors2011language}






\cite{wichmann2011phonological}
\cite{wichmann2008languagephonological}
\cite{smith2010eliminating}
\cite{slobin1982children}

\cite{sapir1912language}
\cite{reali2014paradox}


\cite{lupyanrole}\cite{lupyan2010language}
%This proposal is relatively uncontroversial at the level of semantics \cite{sapir}: Languages have words for meanings that are relevant for their environment (e.g. ). But, at  more abstract levels of language structure, this hypothesis has been very controversial. 
\cite{kirby2008cumulative}
Meaning. \cite{silvey2015word}
\cite{perfors2011language}
Critically, 



%\squeezeup

\begin{figure}[b!]
\begin{center}
\includegraphics[width = 1\linewidth]{figs/}
\end{center}
\vspace{-.24em}
\caption{Object stimuli used in the Experiment. The objects are sorted from least complex (top left) to most complex (bottom right) based on the complexity norms in Lewis et al.\ (2014). Each row corresponds to a quintile.}
\label{fig:objs}
\vspace{-1em}
\end{figure}





\bibliographystyle{apacite}

\setlength{\bibleftmargin}{.125in}
\setlength{\bibindent}{-\bibleftmargin}

\bibliography{biblibrary}
\end{document}
